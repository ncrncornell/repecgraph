% Flowcharting techniques for easy maintenance
% Author: Brent Longborough
% http://www.texample.net/media/tikz/examples/TEX/flexible-flow-chart.tex
% Start the picture
\begin{tikzpicture}[%
    >=triangle 60,              % Nice arrows; your taste may be different
    start chain=going below,    % General flow is top-to-bottom
    node distance=10mm and 60mm, % Global setup of box spacing
    every join/.style={author},   % Default linetype for connecting boxes
    ]
% ------------------------------------------------- 
% A few box styles 
% <on chain> *and* <on grid> reduce the need for manual relative
% positioning of nodes
\tikzset{
  base/.style={draw, on chain, on grid, align=center, minimum height=4ex},
  activity/.style={base, rectangle, text width=8em, minimum height=2em},
  agent/.style={base, diamond, aspect=2, text width=5em},
  entity/.style={base, ellipse,fill=ncentity!25,minimum height=2em,minimum width=8em,align=center},
  legend/.style={base, draw=none,rectangle},
  % coord node style is used for placing corners of connecting lines
  coord/.style={coordinate, on chain, on grid, node distance=6mm and 25mm},
  % nmark node style is used for coordinate debugging marks
  nmark/.style={draw, cyan, circle, font={\sffamily\bfseries}},
  % -------------------------------------------------
  % Connector line styles for different parts of the diagram
  author/.style={->, draw, lcauthor},
  adivsor/.style={->, draw, lcadvisor},
  use/.style={->, draw, lcuse},
  it/.style={font={\small\itshape}}
}
% -------------------------------------------------
% Legend
\node [legend] (l1)             {Role:};
\node [legend, below=of l1, yshift=1.5em ] (l3) {Role:};
\node [legend, below=of l3, yshift=1.5em ] (l5)             {Link:};
\node [legend, below=of l5, yshift=1.5em ] (l7)             {Link:};
\node [legend, right= 30mm of l1] (l2)             {Advisor};
\node [legend, right= 30mm of l3] (l4)             {Author};
\node [legend, right= 30mm of l5] (l6) {Attribution};
\node [legend, right= 30mm of l7] (l8) {Association};
% Start by placing the nodes
\node [entity, below=of l7, densely dotted, it] (e0) {Thesis pab175};
% Use join to connect a node to the previous one 
\node [activity] (a1)     {Ph.D. pab175};
\node [agent] (p2) {pab175};
% We position the next block explicitly as the first block in the
% second column.  The chain 'comes along with us'. The distance
% between columns has already been defined, so we don't need to
% specify it.
\node [agent, fill=lcadvisor!25, right=of a1] (p4) {phe22};
\node [agent, fill=lcadvisor!25] (p5) {pze9};
% -------------------------------------------------
% Now we place the coordinate nodes for the connectors with angles, or
% with annotations. We also mark them for debugging.
%\node [coord, right=20mm of l0] (c0)  {}; \cmark{0}   
%\node [coord, right=10mm of c0] (c1)  {}; \cmark{1}   
%\node [coord, right=20mm of l1] (c2)  {}; \cmark{2}   
%\node [coord, right=10mm of c2] (c3)  {}; \cmark{3}   
% -------------------------------------------------
% A couple of boxes have annotations
%\node [above=0mm of p4, it] {(Queue was empty)};
%\node [above=0mm of p8, it] {(Queue was not empty)};
% -------------------------------------------------
% All the other connections come out of tests and need annotating
% First, the straight north-south connections. In each case, we first
% draw a path with a (consistently positioned) annotation node, then
% we draw the arrow itself.
\path (e0.south) to node [left] {wasGeneratedBy} (a1);
  \draw [->,lcuse] (e0.south) -- (a1);
\path (a1.south) to node [left] {wasAssociatedWith} (p2);
  \draw [->,lcauthor,densely dotted] (a1.south) -- (p2);
\path (a1.east) to node [above] {wasAssociatedWith} (p4);
  \draw [->,lcadvisor,densely dotted] (a1.east) -- (p4);
\path (a1.east) to node [above] {wasAssociatedWith} (p5);
  \draw [->,lcadvisor,densely dotted] (a1.east) -- (p5);
% Legend
% Legend
\path (l1.east) to node {} (l2);
\draw [->,lcauthor,densely dotted] (l1.east) -- (l2);
\path (l3.east) to node {} (l4);
\draw [->,lcattr,densely dotted] (l3.east) -- (l4);
\path (l5.east) to node {} (l6);
\draw [->] (l5.east) -- (l6);
\path (l7.east) to node {} (l8);
\draw [->,densely dotted] (l7.east) -- (l8);

%\path (t2.south) to node [near start, xshift=1em] {$y$} (t3); 
%  \draw [*->,lcauthor] (t2.south) -- (t3);
%\path (t3.south) to node [near start, xshift=1em] {$y$} (t4); 
%  \draw [*->,lcauthor] (t3.south) -- (t4);
%\path (t5.south) to node [near start, xshift=1em] {$y$} (t6); 
%  \draw [*->,lcadivsor] (t5.south) -- (t6);
%\path (t6.south) to node [near start, xshift=1em] {$y$} (t7); 
%  \draw [*->,lcadivsor] (t6.south) -- (t7); 
%% ------------------------------------------------- 
%% Now the straight east-west connections. To provide consistent
%% positioning of the test exit annotations, we have positioned
%% coordinates for the vertical part of the connectors. The annotation
%% text is positioned on a path to the coordinate, and then the whole
%% connector is drawn to its destination box.
%\path (t3.east) to node [near start, yshift=1em] {$n$} (c3); 
%  \draw [o->,lccong] (t3.east) -- (p8);
%\path (t4.east) to node [yshift=-1em] {$k \leq 0$} (c4r); 
%  \draw [o->,lcauthor] (t4.east) -- (p9);
%% -------------------------------------------------
%% Finally, the twisty connectors. Again, we place the annotation
%% first, then draw the connector
%\path (t1.east) to node [near start, yshift=1em] {$n$} (c1); 
%  \draw [o->,lcadivsor] (t1.east) -- (c1) |- (p4);
%\path (t2.east) -| node [very near start, yshift=1em] {$n$} (c1); 
%  \draw [o->,lcadivsor] (t2.east) -| (c1);
%\path (t4.west) to node [yshift=-1em] {$k>0$} (c4); 
%  \draw [*->,lcauthor] (t4.west) -- (c4) |- (p3);
%\path (t5.east) -| node [very near start, yshift=1em] {$n$} (c6); 
%  \draw [o->,lcadivsor] (t5.east) -| (c6); 
%\path (t6.east) to node [near start, yshift=1em] {$n$} (c6); 
%  \draw [o->,lcadivsor] (t6.east) -| (c7); 
%\path (t7.east) to node [yshift=-1em] {$k \leq 0$} (c7); 
%  \draw [o->,lcadivsor] (t7.east) -- (c7)  |- (p9);
%\path (t7.west) to node [yshift=-1em] {$k>0$} (c5); 
%  \draw [*->,lcadivsor] (t7.west) -- (c5) |- (p5);
% -------------------------------------------------
% -------------------------------------------------
\end{tikzpicture}
